\documentclass[11pt,a4paper]{article}

\usepackage[utf8]{inputenc}
\usepackage[russian,english]{babel}
\usepackage{amsmath,amssymb,amsthm}
\usepackage{hyperref}
\usepackage{geometry}
\geometry{margin=2.5cm}

\title{GRA-Multiverse-Optimizer:\\
A Prototype Backend for Multilevel GRA Meta-Obnulyonka in a Multiverse of Subsystems}

\author{Your Name\\
\small Independent researcher\\
\small \texttt{you@example.com}}

\date{2026}

\begin{document}

\maketitle

\begin{abstract}
\noindent
\textbf{EN.}
We present \emph{GRA-Multiverse-Optimizer}, a prototype backend for the multilevel GRA Meta-Obnulyonka ``multiverse optimizer''. The framework represents a multiverse state \(\mathbf{\Psi}\) as a collection of subsystems across several abstraction levels, defines a multilevel functional \(J_{\text{multiverse}}(\mathbf{\Psi})\) with level-dependent ``foam'' terms \(\Phi^{(l)}\), and implements simple gradient-based procedures to approximate multiverse nulling. On top of this backend we demonstrate toy applications such as LLM answer alignment (anti-hallucination) and VPN configuration selection, illustrating how the theory can guide practical architecture and configuration evolution.

\medskip
\noindent
\textbf{RU.}
Мы представляем \emph{GRA-Multiverse-Optimizer} --- прототип бэкенда для многоуровневой GRA Мета-обнулёнки в виде ``мультиверса'' подсистем. В рамках подхода мультиверсное состояние \(\mathbf{\Psi}\) задаётся как набор подсистем на нескольких уровнях абстракции, определяется многоуровневый функционал \(J_{\text{multiverse}}(\mathbf{\Psi})\) с пенной \(\Phi^{(l)}\) на каждом уровне и реализуются простые градиентные процедуры приближённого мультиверсного обнуления. Поверх этого бэкенда демонстрируются игрушечные примеры (согласование ответов LLM, выбор VPN-конфигураций), показывающие, как теория может направлять эволюцию архитектур и конфигураций на практике.
\end{abstract}

\section{Introduction}

Recent progress in large language models and multi-agent systems motivates architectures that can reason over many subsystems and domains simultaneously. The GRA Meta-Obnulyonka framework proposes a multilevel ``nulling'' principle, where cognitive foam is minimized across an entire hierarchy of goals and abstractions. In this work we provide a minimal numerical backend --- \emph{GRA-Multiverse-Optimizer} --- and a practical software artefact suitable for citation and reproduction via Zenodo and GitHub.

\section{Multiverse state and functional}

We consider a hierarchy of levels \(l = 0,\dots,K\). Each subsystem is indexed by a multi-index
\[
\mathbf{a} = (a_0, a_1, \dots, a_k),
\]
and has a Hilbert space \(\mathcal{H}^{(\mathbf{a})}\). The level-\(l\) state space is
\[
\mathcal{H}^{(l)} =
\bigotimes_{\mathbf{a}:\,\dim(\mathbf{a})=l}
\mathcal{H}^{(\mathbf{a})},
\]
and the full multiverse space is
\[
\mathcal{H}_{\text{multiverse}} =
\bigotimes_{l=0}^K \mathcal{H}^{(l)}.
\]

For each level \(l\) we define a goal \(G_l\) and a projector \(\mathcal{P}_{G_l}\). The foam term at level \(l\) is
\[
\Phi^{(l)}(\Psi^{(l)}, G_l) =
\sum_{\mathbf{a}\neq\mathbf{b} \atop
      \dim(\mathbf{a})=\dim(\mathbf{b})=l}
\big|\langle \Psi^{(\mathbf{a})} \mid
         \mathcal{P}_{G_l} \mid
         \Psi^{(\mathbf{b})} \rangle\big|^2.
\]

The global multiverse state is
\[
\mathbf{\Psi} = \{\Psi^{(\mathbf{a})}\}_{\mathbf{a} \in \mathcal{I}},
\]
and the multiverse functional is given by
\[
J_{\text{multiverse}}(\mathbf{\Psi}) =
\sum_{l=0}^K \Lambda_l
\sum_{\dim(\mathbf{a})=l}
J^{(l)}(\Psi^{(\mathbf{a})}),
\]
where \(J^{(l)}\) is defined recursively in terms of local losses and foam.

\section{Prototype implementation}

The repository \texttt{GRA-Multiverse-Optimizer} provides:

\begin{itemize}
  \item core data structures:
    \texttt{Level}, \texttt{Goal}, \texttt{MultiverseState},
    \texttt{MultiverseFunctional};
  \item a simple gradient-based optimizer \texttt{MultiverseOptimizer}
    using finite-difference gradients;
  \item helper modules:
    \texttt{llm\_module.py} (LLM answer alignment) and
    \texttt{vpn\_module.py} (VPN configuration selection);
  \item example Jupyter notebooks and unit tests.
\end{itemize}

These components form a minimal backend to experiment with multiverse nulling and architecture evolution.

\section{Applications and outlook}

The LLM example demonstrates how several conflicting answers can be embedded, interpreted as level-0 subsystems, coupled via a level-1 meta-node, and optimized to select a consensus answer with minimal multiverse foam. The VPN example treats different network configurations as level-0 subsystems and uses a meta-goal to favour stable, stealthy configurations.

Future work includes:
\begin{itemize}
  \item replacing finite-difference gradients with analytic or autograd-based ones,
  \item implementing an explicit duplication--specialization operator \(\mathcal{D}\) for architecture evolution,
  \item connecting the numerical backend to full GRA Meta-Obnulyonka theory and AGI/ASI research.
\end{itemize}

\section*{Acknowledgements}

The author thanks the broader AGI and open-source communities for inspiration and feedback.

\bibliographystyle{plain}
% You can add a small .bib file if needed, or keep this empty for Zenodo.

\end{document}
